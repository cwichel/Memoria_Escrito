% =========================================================
% CONFIGURACION DEL DOCUMENTO
% =========================================================
\providecommand{\main}{..}
\documentclass[\main/main.tex]{subfiles}

% =========================================================
% CONTENIDO
% =========================================================
\begin{document}
\chapter{Resumen}
	En el presente trabajo de título se implementará un sistema de estimulación visual y registro de movimiento ocular para ser utilizado en experimentos que involucran tareas sicomotoras. 
	
	En la primera parte del escrito se presenta en líneas generales los principios del movimiento ocular y algunas de las tecnologías utilizadas en la actualidad tanto para estimular como registrar dichos movimientos, además de presentar una configuración de espacio de trabajo típica para realizar este tipo de tareas.

	La segunda parte del escrito consiste en mostrar como, en base a los requerimientos de los experimentos a implementar, se plantea y construye el sistema. Dichos requerimientos dicen relación tanto con las características técnicas de las tareas como con la necesidad de registrar correctamente los parámetros de configuración asociadas a las mismas. 

	Finalmente se presenta, a modo de manual de usuario, los pasos a seguir para montar un experimento tipo ya sea mediante código o la interfaz gráfica y se muestran algunos de los resultados obtenidos con el fin de mostrar su correcto funcionamiento.
	
% \chapter{Abstract}	
	
\end{document}
	