% =========================================================
% CONFIGURACION DEL DOCUMENTO
% =========================================================
\providecommand{\main}{..}
\documentclass[\main/main.tex]{subfiles}

% =========================================================
% CONTENIDO
% =========================================================
\begin{document}
\chapter{Resumen}
Este trabajo de titulación contiene los procedimientos necesarios para el modelado, diseño, simulación e implementación parcial de un sistema de control embebido para un vehículo aéreo no tripulado de cuatro hélices “Quadcopter”.
\hfill \break
En Capítulo 1: “Introducción y Objetivos”, se tiene una introducción al tema, el estado actual de investigación relacionado al proyecto y los objetivos a alcanzar en el desarrollo de este mismo.
\hfill \break
En Capítulo 2: “Modelado de la Planta”, se presenta un estudio de las dinámicas asociadas al sistema, siguiendo dos enfoques diferentes que convergen a un mismo modelo. 
\hfill \break
En Capítulo 3: “Diseño de Controlador”, se realiza una linealización del sistema con el fin de abordar el problema con herramientas de control lineal, como lo son la familia de controladores PID. También se diseña un control a implementar en simulación mediante LGR.
\hfill \break
En Capítulo 4: “Caracterización y Simulación”, se presentan las mediciones y cálculos realizados para la caracterización de actuadores y la inercia de la planta. Posteriormente se muestran los resultados obtenidos de la simulación con el control diseñado en el capitulo 3.
\hfill \break
En Capítulo 5: “Componentes para la Implementación”, se estudian las principales características de los componentes a utilizar para el control embebido, tales como controlador, sensores, actuadores, soportes y baterías.
\hfill \break
En Capítulo 6: “Implementación en la Planta Real”, se presentan los resultados obtenidos de las pruebas experimentales con la planta real y los dos soportes diseñados. 
\hfill \break
En Capítulo 7: “Conclusiones y Trabajos Futuros”, como su nombre lo señala, presenta todo aquello que se extrae a partir de este trabajo de título y las futuras mejoras que se pueden implementar.
\newpage
	
\chapter{Abstract}	
This work contains the procedures necessaries for modeling, design, simulation and partial implementation of an embedded control system for four-helix unmanned aerial vehicle "Quadcopter".
\hfill \break
Chapter 1: “Introduction and Objectives", presents an introduction to the current state of research related to the project and the objectives to be achieved in the development of this.
\hfill \break
Chapter 2: “Modeling Plant", presents a study of the dynamics associated to the system, by analizing two different approaches that coverge to the same model.
\hfill \break
Chapter 3: “Designing Control Law", a linearization of the system is done in order to solve the problem with linear control strategies, as are the family of PID controllers. Also a LGR control is desing to implement in simulation. 
\hfill \break
Chapter 4: “Characterization and Simulation", presents measurements and calculations performed for the actuators characterization and inertia. Also pesents the results of the simulation with the control law desingned in chapter 3.
\hfill \break
Chapter 5: “Components for Implementation", the main characteristics of the components are studied to be used for embedded, such as controller, sensors, actuators, supports and battery.
\hfill \break
Chapter 6: “Implementation in the Real Plant", the results of experimental tests with the real plant and two holders designed are presented.
\hfill \break
Chapter 7: “Conclusions and Future Work" as its name indicates, presents all that is extracted from the title work and future improvements that can be implemented.
	
\end{document}
	