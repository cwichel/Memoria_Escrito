% =========================================================
% CONFIGURACION DEL DOCUMENTO
% =========================================================
\providecommand{\main}{..}
\documentclass[\main/Main.tex]{subfiles}

% =========================================================
% CONTENIDO
% =========================================================
\begin{document}
\chapter{Resumen}
	Este trabajo de título muestra la construcción e implementación de un sistema de estimulación visual y registro de movimiento ocular para uso en experimentos que evalúan tareas sicomotoras.

	El estudio de las dinámicas del ojo y la capacidad de registrar sus movimientos ha permitido avances importantes en áreas diversas como entender el desarrollo de la enfermedad de Parkinson, estudiar la usabilidad de productos o la efectividad de la publicidad o entrenamiento en sistemas simulados, entre otros. 

	Cada experimento tiene requerimientos específicos, relacionados a las características técnicas de las tareas y a la forma en que se registran los parámetros de configuración de las mismas. El sistema desarrollado en este trabajo de título permite al usuario montar un experimento, ingresar los parámetros y definir tareas a través de una interfaz gráfica, sin requerir alterar el código del software. 

	El sistema propuesto incluye hardware (PC, monitores, \textit{eye tracker} y apoya barbilla) y software para la adquisición de datos y la proyección del experimento.

	Para demostrar el funcionamiento del sistema se incluye un ejemplo de aplicación, con instrucciones para realizar un experimento de muestra, ya sea mediante código o usando la interfaz gráfica, además de mostrar los resultados de este experimento.

	
\chapter{Abstract}
	This final project shows the construction and implementation of a visual stimulation system and an ocular movement record to use in experiments that evaluate psychomotor tasks.
	
	The study of eye dynamics and the capacity to record it's movements has allowed important advances in diverse areas such as undestanding the development of the Parkinson's decease, study the usability of products or the effectivity of advertisment or training in simulate systems, among others. 
	
	Each experiment have different requirements related to the technical characteristics of the tasks and the way in which the configuration parameters are recorded. The system developed in this final project allows the user to setup an experiment, enter the parameters and define tasks through a graphical user interface without altering the code or the software.
	
	The proposed system includes hardware (PC, monitors, eye trackers and chin rest) and software for the acquisition of data and the proyection of the experiment.
	
	To show how the system works an application example is included with instructions to run a sample experiment either through code or using the graphical user interface as well as showing the results of this experiment. 
	 
\end{document}
	