% =========================================================
% CONFIGURACION DEL DOCUMENTO
% =========================================================
\providecommand{\main}{..}
\documentclass[\main/main.tex]{subfiles}

% =========================================================
% CONTENIDO
% =========================================================
\begin{document}
\chapter{Resumen}
	Este trabajo de título muestra la construcción e implementación de un sistema de estimulación visual y registro de movimiento ocular para uso en experimentos que evalúan tareas sicomotoras.

	El estudio de las dinámicas del ojo y la capacidad de registrar sus movimientos ha permitido avances importantes en áreas diversas como entender el desarrollo de la enfermedad de Parkinson, estudiar la usabilidad de productos o la efectividad de la publicidad o entrenamiento en sistemas simulados, entre otros. 

	Cada experimento tiene requerimientos específicos, relacionados a las características técnicas de las tareas y a la forma en que se registran los parámetros de configuración de las mismas. El sistema desarrollado en este trabajo de título permite al usuario montar un experimento, ingresar los parámetros y definir tareas a través de una interfaz gráfica, sin requerir alterar el código del software. 

	El sistema propuesto incluye hardware (PC, monitores, \textit{eye tracker} y apoya barbilla) y software para la adquisición de datos y la proyección del experimento.

	Para demostrar el funcionamiento del sistema se incluye un ejemplo de aplicación, con instrucciones para realizar un experimento de muestra, ya sea mediante código o usando la interfaz gráfica, además de mostrar los resultados de este experimento.

	
% \chapter{Abstract}	
	
\end{document}
	