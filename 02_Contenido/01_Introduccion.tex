% =========================================================
% CONFIGURACION DEL DOCUMENTO
% =========================================================
\providecommand{\main}{..}
\documentclass[../main.tex]{subfiles}

% =========================================================
% CONTENIDO
% =========================================================
\begin{document}		
\chapter{Introducción}
\label{cha:01_introducción}
	\section{Introducción}
	\label{sec:01_introducción}
		Todos los días usamos nuestros ojos para relacionarnos con nuestro ambiente: leemos un libro, vemos la televisión, usamos el computador, miramos el paiseaje, etc. Desconociendo por completo la complejidad asociada al proceso de adquisición de imágenes por nuestro sistema visual. El ojo debe realizar... 

		Los sistemas de eye tracking son utilizados...

		Las oportunidades que esto conlleva...


	\section{Motivación y objetivos}
	\label{sec:01_motivacón_y_objetivos}
		Los estudios sobre movimiento ocular permiten...

		Lo que hace pensar que...

		Por esto, se pretende...

		Así, el objetivo principal de este trabajo de título consiste en el diseño y construcción de un sistema de estimulación visual y registro de movimientos oculares para tareas sicomotoras con el fin de facilitar el proceso de puesta en marcha de experimentos asociados al estudio de... . Con lo cual se hace necesario:
		\singlespacing{\begin{enumerate}[(i)]
			\item Definir y programar el mecanismo de estimulación visual en acorde con las características técnicas del despliegue.
			\item Diseñar e implementar un sistema de sinconización entre el registro y la estimulación.
			\item Integrar al diseño un sistema de captura del movimiento ocular.
			\item Asegurar el correcto registro de los datos.
			\item Construcción de protocolos de estimulación para tareas sicomotoras.
			\item Integración de todas las etapas en una GUI. 	
		\end{enumerate}}

\end{document}
