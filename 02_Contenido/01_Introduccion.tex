% =========================================================
% CONFIGURACION DEL DOCUMENTO
% =========================================================
\providecommand{\main}{..}
\documentclass[../Main.tex]{subfiles}

% =========================================================
% CONTENIDO
% =========================================================
\begin{document}		
\chapter{Introducción}
\label{cha:01_introducción}
	% ===============================================================
	% ===============================================================
	\section{Motivación}
	\label{sec:01_motivacion}
		Diariamente y sin prestar mayor atención, gran parte de la población utiliza sus ojos para interactuar con su entorno: se detienen a admirar el paisaje, leer un libro, revisar su teléfono, navegar en internet, verificar que sus alimentos se encuentran en buen estado, etc. A pesar de lo simple que puede parecer esta acción, en la realidad corresponde a un proceso sumamente complejo que involucra la participación de un sinnúmero de estructuras sensoriomotrices. 

		El simple hecho de orientar nuestra vista hacia un nuevo objetivo desencadena una serie de eventos fascinantes: El globo ocular rota hasta lograr posicionarse en una determinada dirección de forma tal que los rayos de luz que son reflejados por el punto de interés se proyectan en la \gls{retina}, estructura que transduce esta información en impulsos eléctricos que son interpretados de forma posterior por el cerebro y que se traducen en información que percibimos como una imagen.  

		El estudio de las dinámicas del ojo y la capacidad de registrar sus movimientos ha permitido con el paso de los años avances importantes en áreas sumamente variadas como la detección y seguimiento de enfermedades, estudios de marketing y usabilidad, ayuda a personas con discapacidad, entrenamiento y detección de errores en sistemas simulados, defensa y seguridad, videojuegos y experiencia de usuario, entre otras.

		La principal dificultad en los avances de todas las áreas recae no solo en la calidad del registro ocular tanto en precisión temporal como espacial, si no, en la capacidad de correlacionar estímulos visuales y respuestas motrices con el fin de detectar la relación causa/efecto de los eventos. Para lograr esto, es necesario el uso de sistemas que permitan sincronizar las etapas de estimulación y registro de respuesta de forma tal de entregar insumos que permitan la obtención de información pertinente y útil del procesamiento posterior.  

	% ===============================================================
	% ===============================================================
	\section{Objetivos}
	\label{sec:01_objetivos}
		Así, el objetivo principal de este trabajo de título consiste en el diseño y construcción de un sistema de estimulación visual y registro de movimientos oculares enfocado en la realización de experimentos asociados a tareas sicomotoras. De esta forma, se presentan a continuación los objetivos específicos para el desarrollo: 
		
		\begin{enumerate}[(I)]\setlength\itemsep{-0.5em}
			\item Definir y programar un mecanismo de estimulación visual acorde con las características técnicas de los protocolos utilizados en la realización de tareas sicomotoras.
			
			\item Implementar un sistema de sincronización entre el registro y la estimulación visual que permita la integración de un sistema de captura de movimiento ocular y que asegure el correcto registro de los datos.
			
			\item Integrar todas las etapas en una interfaz gráfica de usuario (\textit{\acrshort{gui}}). 	
		
		\end{enumerate}

	% ===============================================================
	% ===============================================================
\end{document}
