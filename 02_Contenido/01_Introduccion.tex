% =========================================================
% CONFIGURACION DEL DOCUMENTO
% =========================================================
\providecommand{\main}{..}
\documentclass[../main.tex]{subfiles}

% =========================================================
% CONTENIDO
% =========================================================
\begin{document}		
\chapter{Introducción}
\label{cha:01_introducción}
	\section{Motivación}
	\label{sec:01_motivacion}

		Diariamente y sin prestar mayor atención la gran parte parte de la población utiliza sus ojos para interactuar con su entorno: se detienen a admirar el paisaje, leer un libro, revisar su teléfono, navegar en internet, verificar que sus alimentos se encuentran en buen estado, etc. A pesar de lo simple que puede parecer esta acción, en la realidad corresponde a un proceso sumamente complejo que involucra la participación de un sinnúmero de estructuras del cuerpo. 

		El simple hecho de orientar nuestra vista hacia un nuevo objetivo desencadena una serie de eventos fascinantes: El globo ocular rota hasta lograr posicionarse en una determinada dirección de forma tal que los rayos de luz que son reflejados por el punto de interés se proyectan en la córnea, estructura que transduce esta información en impulsos eléctricos que son interpretados de forma posterior por el cerebro y que se traducen en información que percibimos como una imagen. 

		El estudio de las dinámicas del ojo y la capacidad de registrar sus movimientos ha permitido con el paso de los años avances importantes en áreas sumamente variadas como la detección y seguimiento de enfermedades, estudios de marketing y usabilidad, ayuda a personas con dispacidad, entrenamiento y detección de errores en sistemas simulados, defensa y seguridad, videojuegos y experiencia de usuario, entre otras.

		El factor común de todos estos avances recae no solo en la calidad del registro ocular, si no, en la capacidad de correlacionar estímulos visuales y respuestas en forma de actividad motriz. Esto hace fundamental el desarrollo de sistemas que permitan sincronizar estimulación/respuesta de forma tal de entregar insumos de calidad que permitan la obtención de información pertinente y útil del procesamiento posterior.  

	\section{Objetivos}
	\label{sec:01_objetivos}

		Así, el objetivo principal de este trabajo de título consiste en el diseño y construcción de un sistema de estimulación visual y registro de movimientos oculares que se enfocará a la realización de experimentos asociados a tareas sicomotoras con el fin de facilitar el proceso de puesta en marcha de experimentos para estudios en neurociencia. De esta forma se presentan a continuación las etapas propuestas para el desarrollo: 
		
		\singlespacing{\begin{enumerate}[(i)]
			\item Definir y programar el mecanismo de estimulación visual acorde con las características técnicas del despliegue.
			\item Diseñar e implementar un sistema de sinconización entre el registro y la estimulación.
			\item Integrar al diseño un sistema de captura del movimiento ocular.
			\item Asegurar el correcto registro de los datos.
			\item Construcción de protocolos de estimulación para tareas sicomotoras.
			\item Integración de todas las etapas en una \acrshort{gui}. 	
		\end{enumerate}}

\end{document}
