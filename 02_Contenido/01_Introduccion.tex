% =========================================================
% CONFIGURACION DEL DOCUMENTO
% =========================================================
\providecommand{\main}{..}
\documentclass[../main.tex]{subfiles}

% =========================================================
% CONTENIDO
% =========================================================
\begin{document}		
\chapter{Introducción}
\label{cha:01_introducción}
	\section{Motivación}
	\label{sec:01_motivacion}

		Diariamente y sin prestar mayor atención la mayor parte de la población utiliza sus ojos para interactuar con su entorno: se detienen a admirar el paisaje, leer un libro, revisar su teléfono, navegar en internet, verificar que sus alimentos se encuentran en buen estado, etc. A pesar de lo simple que puede parecer esta acción, en la realidad corresponde a un proceso sumamente complejo que involucra la participación de un sinnúmero de estructuras del cuerpo. 

		El simple hecho de orientar nuestra vista hacia un nuevo objetivo desencadena una serie de eventos fascinantes: El globo ocular rota hasta lograr posicionarse en una determinada dirección de forma tal que los rayos de luz que son reflejados por el punto de interés se proyectan en la córnea, estructura que transduce esta información en impulsos eléctricos que son interpretados de forma posterior por el cerebro y que se traducen en información que interpretamos como una imagen. 

		El estudio de las dinámicas del ojo, incluyendo su movimiento (horizontal, vertical y rotacional) y el comportamiento de la pupila ha dado paso a una serie de avances que van desde detección de enfermedades (por ejemplo) hasta estudios de marketing y usabilidad (por ejemplo ). El elemento común entre estas aplicaciones tan diversas corresponde a sistemas confiables que permitan correlacionar los diversos estímulos visuales con las respuestas motoras obtenidas.

		Las aplicaciones más llamativas, no obstante, corresponden a la detección temprana de  


		Aplicación particular: eye tracking en monitorizacion de funciones cognitivas

		- Motivación -> Posible aplicación: Detección temprana de enfermedades neurológicas: Demencia? Parkinson?  


	\section{Motivación y objetivos}
	\label{sec:01_motivacón_y_objetivos}
		Los estudios sobre movimiento ocular permiten...

		Lo que hace pensar que...

		Por esto, se pretende...

		Así, el objetivo principal de este trabajo de título consiste en el diseño y construcción de un sistema de estimulación visual y registro de movimientos oculares para tareas sicomotoras con el fin de facilitar el proceso de puesta en marcha de experimentos asociados al estudio de... . Con lo cual se hace necesario:
		
		\singlespacing{\begin{enumerate}[(i)]
			\item Definir y programar el mecanismo de estimulación visual en acorde con las características técnicas del despliegue.
			\item Diseñar e implementar un sistema de sinconización entre el registro y la estimulación.
			\item Integrar al diseño un sistema de captura del movimiento ocular.
			\item Asegurar el correcto registro de los datos.
			\item Construcción de protocolos de estimulación para tareas sicomotoras.
			\item Integración de todas las etapas en una GUI. 	
		\end{enumerate}}

\end{document}
