% =========================================================
% CONFIGURACION DEL DOCUMENTO
% =========================================================
\providecommand{\main}{..}
\documentclass[\main/main.tex]{subfiles}

% =========================================================
% CONTENIDO
% =========================================================
\begin{document}
\chapter{Resultados}
\label{cha:04_resultados}
	
	


	\begin{python}[caption=Crear la carpeta del experimento y la base de datos., captionpos=b, label=alg:04_new_db]
	import os
    from saccadeApp import SaccadeDB, Master
    from saccadeApp import Experiment, Test, Frame, Component

    # Se crean los directorios para almacenar las configuraciones y resultados de los experimentos
    base_dir = u'C:\\Experiments'
    data_dir = base_dir + u'\\Experiment_Data'
    if not os.path.isdir(base_dir):
        os.mkdir(base_dir)
        os.mkdir(data_dir)
    elif not os.path.isdir(data_dir):
        os.mkdir(data_dir)

    # Se abre (o si no existe, crea) la base de datos local
    data = SaccadeDB(base_dir + u'\\test_database.sqlite3')

	\end{python}

	\begin{python}[caption=Comprobar las configuraciones de los monitores., captionpos=b, label=alg:04_check_monitor]
	# Se abre la instancia de configuración de monitores para verificar que el perfil a ser escogido existe
    Master.open_psychopy_monitor_center()

	\end{python}


	\begin{python}[caption=Crear un nuevo perfil de configuración., captionpos=b, label=alg:04_new_conf]
	# Se crea una instancia de configuración
    conf = Master()
    conf.set_database(data)

    # Se asigna nombre, se selecciona la pantalla secundaria con el perfil 'test_profile' para un eyetracker
    # 'eyetribe'. Los resultados de los experimentos se almacenarán en el directorio de datos
    conf.set_name(u'test_profile')
    conf.set_screen(1)
    conf.set_monitor(u'test_monitor')
    conf.set_tracker(u'eyetribe')
    conf.set_experiment_path(data_dir)

    # Y finalmente se guarda la configuración
    conf.save()

	\end{python}


	\begin{python}[caption=Crear un nuevo experimento., captionpos=b, label=alg:04_new_exp]
	# Construccion del primer test: Movimiento pro/anti-sacádico  con overlap
    # Componentes
    cross = Component()
    cross.set_size(0.5)
    cross.set_shape(u'cross')
    cross.set_position(0.0, 0.0)

    target = Component()
    target.set_size(0.5)
    target.set_shape(u'square')
    target.set_position(16.0, 0)

    # Cuadros
    frame1 = Frame()
    frame1.set_name(u'Central Stimulus')
    frame1.set_time(1.8)
    frame1.set_as_task(False)
    frame1.set_color(u'black')
    frame1.component_add(cross)

    frame2 = frame1.copy()
    frame2.set_name(u'Target')
    frame2.set_time(1.5)
    frame2.component_add(target)

    frame3 = frame1.copy()
    frame3.set_time(1.0)

    # Tarea
    test1 = Test()
    test1.set_name(u'Overlap task')
    test1.set_quantity(1)
    test1.frame_add(frame1)
    test1.frame_add(frame2)
    test1.frame_add(frame3)

    # Construccion del segundo test: Patrones de busqueda
    # Componente
    image = Component()
    image.set_image(u'couple.png')

    # Cuadro
    frame4 = Frame()
    frame4.set_name(u'Image Presentation')
    frame4.set_as_task(True)
    frame4.set_keys_allowed(u'space')
    frame4.set_keys_selected(u'space')
    frame4.component_add(image)

    # Tarea
    test2 = Test()
    test2.set_name(u'Search task')
    test2.set_quantity(1)
    test2.frame_add(frame4)

    # Se construye el experimento
    experiment = Experiment()
    experiment.set_database(data)
    experiment.set_code(u'expcode_01')
    experiment.set_info(u'Experiment test', u'v1.0')
    experiment.set_space_start(True)
    experiment.set_dialog(status=False)
    experiment.test_add(test1)
    experiment.test_add(test2)

    # Y finalmente se guarda la configuración
    experiment.save()

	\end{python}

	
	\begin{python}[caption=Crear un nuevo experimento., captionpos=b, label=alg:04_new_exp]
	from saccadeApp import SaccadeDB
	from saccadeApp import ExperimentHandler

	# Se abre la base de datos local
    data = SaccadeDB(base_dir + u'\\test_database.sqlite3')

    # Con las configuraciones guardadas es posible ejecutar el experimento
    execution = ExperimentHandler()

    if execution.load_experiment(db=data, mas=u'test_profile', exp=u'expcode_01'):
        execution.save_parameters()
        execution.execute_experiment()

	\end{python}


	respecto del objetivo:
		1) Es posible configurar y almacenar [referencia a código de ejemplo de configuración y screenshot de los datos almacenados en la tabla]:
			- Configuración del sistema en donde se ejecutará el experimento :
				* Eye tracker a ser utilizado.
				* Qué monitor será utilizado y cuál es su perfil de configuración.
				* Directorio donde se almacenarán los resultados del experimento.
			- Configuración de los experimentos:
				* Configurar sus tareas, cuadros y componentes.
				* Opciones de ejecución:
					+ Tiempos de descanso.
					+ Tipo de ejecución de las tareas: aleatoria o secuencial.
					+ Si se requiere alguna acción del usuario para comenzar cada tarea o la ejecución será ininterrumpida.
				* Opciones asociadas al diálogo previo a cada ejecución (sesión). Esto dice relación con la información que interesa rescatar del paciente que realizará el experimento.

		2) Es posible ejeutar el experimento de forma tal que se almacene tanto la información del estímulo presentado como de las respuestas producidas por el usuario [screenshot del árbol de capetas producido dado la configuración anterior junto a imágenes del archivo de resultados y el contenido de los diccionarios generados].
			* Se almacena un respaldo de la configuración utilizada en cada sesión en un archivo de diccionario de formato yaml, lo que facilita su lectura por parte de los investigadores.
			* Se almacena un registro de los eventos ocurridos duraante el laboratorio en un archivo de formato estándar (hdf5) para:
				+ Eye tracker.
				+ Acciones de teclado.  
				+ Eventos en el monitor dependientes de los estímulos presentados.

	% ===============================================================
	% ===============================================================
\end{document}