% =========================================================
% CONFIGURACION DEL DOCUMENTO
% =========================================================
\providecommand{\main}{..}
\documentclass[\main/main.tex]{subfiles}

% =========================================================
% CONTENIDO
% =========================================================
\begin{document}
\chapter{Resultados}
\label{cha:04_resultados}
	respecto del objetivo>:
		1) Es posible configurar y almacenar [referencia a código de ejemplo de configuración y screenshot de los datos almacenados en la tabla]:
			- Configuración del sistema en donde se ejecutará el experimento :
				* Eye tracker a ser utilizado.
				* Qué monitor será utilizado y cuál es su perfil de configuración.
				* Directorio donde se almacenarán los resultados del experimento.
			- Configuración de los experimentos:
				* Configurar sus tareas, cuadros y componentes.
				* Opciones de ejecución:
					+ Tiempos de descanso.
					+ Tipo de ejecución de las tareas: aleatoria o secuencial.
					+ Si se requiere alguna acción del usuario para comenzar cada tarea o la ejecución será ininterrumpida.
				* Opciones asociadas al diálogo previo a cada ejecución (sesión). Esto dice relación con la información que interesa rescatar del paciente que realizará el experimento.

		2) Es posible ejeutar el experimento de forma tal que se almacene tanto la información del estímulo presentado como de las respuestas producidas por el usuario [screenshot del árbol de capetas producido dado la configuración anterior junto a imágenes del archivo de resultados y el contenido de los diccionarios generados].
			* Se almacena un respaldo de la configuración utilizada en cada sesión en un archivo de diccionario de formato yaml, lo que facilita su lectura por parte de los investigadores.
			* Se almacena un registro de los eventos ocurridos duraante el laboratorio en un archivo de formato estándar (hdf5) para:
				+ Eye tracker.
				+ Acciones de teclado.  
				+ Eventos en el monitor dependientes de los estímulos presentados.

	% ===============================================================
	% ===============================================================
\end{document}