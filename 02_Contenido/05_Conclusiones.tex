% =========================================================
% CONFIGURACION DEL DOCUMENTO
% =========================================================
\providecommand{\main}{..}
\documentclass[\main/Main.tex]{subfiles}

% =========================================================
% CONTENIDO
% =========================================================
\begin{document}
\chapter{Conclusiones y trabajo futuro}
\label{cha:05_conclusiones_y_trabajo_futuro}
	% ===============================================================
	% ===============================================================
	\section{Conclusiones}
	\label{sec:05_conclusiones}
		En este trabajo de título fue posible construir un sistema de estimulación visual y registro ocular para la realización de experimentos que evalúan tareas sicomotoras que cumple con los objetivos propuestos al ser capaz de:
		\begin{enumerate}\setlength\itemsep{-0.2em}
			\item Permitir la configuración de experimentos y su posterior presentación.
			\item Permitir sincronizar los eventos de estimulación y registro de movimiento ocular, almacenando los datos obtenidos en un archivo.

		\end{enumerate} 

		No obstante lo anterior y a pesar de que el sistema es capaz de montar y ejecutar los experimentos propuestos (para movimiento pro/anti-sacádico y ejercicios de memoria), presenta limitaciones para la realización de tareas mas complejas o añadir variaciones aleatorias de forma dinámica en el experimento, como cambiar los tiempos en que se presenta un cuadro durante la ejecución, lo que limita cierto tipo de análisis. 
 		 

	% ===============================================================
	% ===============================================================
	\newpage
	\section{Trabajo futuro}
	\label{sec:05_trabajo_futuro}
		Como trabajo para futuros desarrollos se proponen las siguientes tareas:
		\begin{enumerate}\setlength\itemsep{-0.2em}
			\item Añadir la opción de que un cuadro temporizado tenga tiempo variable dentro de un rango definido.
			\item Añadir dentro de una tarea conjuntos de cuadros de selección aleatoria con el fin de optimizar el proceso de configuración. Eso quiere decir que tareas que tengan diferencias mínimas pueden ser almacenadas y posteriormente revisadas de forma más eficiente. Por ejemplo: Tarea de movimiento pro-sacádico en distintas direcciones o con distintas distancias al estímulo central.  
			\item Implementar una mayor variedad de opciones para la configuración de componentes de cuadro, ya sea añadiendo nuevos elementos gráficos como la posibilidad de incluir estímulos sonoros.  
			\item Agregar nuevos elementos al \textit{script} para mejorar y complementar el sistema de registro de eventos en el archivo de salida. Algunas opciones son:
				\begin{enumerate}\setlength\itemsep{-0.2em}
					\item Agregar una lista de los componentes añadidos para cada cuadro con el fin de verificar si fueron cargados correctamente.
					\item Modificar la tabla de registro de eventos de teclado para agregar información sobre si las teclas presionadas corresponden a las requeridas para una tarea específica. Una forma de realizar esto es con un \textit{flag} que indique si la elección fue correcta. 
					\item Para el caso en que el usuario detiene una sesión durante la ejecución: Incluir una ventana de diálogo para que pueda ser ingresado el motivo de la detención, agregando esta información al registro. 

				\end{enumerate}
			\item Incluir una mayor variedad de preguntas de usuario para el inicio de sesión de un experimento. 
			\item Incluir instancias para la pre-visualización de cuadros individuales y tareas.

		\end{enumerate}
	
	% ===============================================================
	% ===============================================================
\end{document}
