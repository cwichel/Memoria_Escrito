% =========================================================
% CONFIGURACION DEL DOCUMENTO
% =========================================================
\providecommand{\main}{..}
\documentclass[\main/main.tex]{subfiles}

% =========================================================
% CONTENIDO
% =========================================================
\begin{document}
\chapter{Conclusiones y trabajo futuro}
\label{cha:05_conclusiones_y_trabajo_futuro}
	\section{Conclusiones}
	\label{sec:05_conclusiones}
		En base a los avances realizados en este proyecto de título se cumplen los objetivos indicados a continuación: 
		\begin{enumerate}[(i)]\setlength\itemsep{-0.2em}
			\item Definir y programar un mecanismo de estimulación visual acorde con las características técnicas de los protocolos utilizados en la realización de tareas sicomotoras.
			
			\item Implementar un sistema de sincronización entre el registro y la estimulación visual que permita la integración de un sistema de captura de movimiento ocular y que asegure el correcto registro de los datos.

		\end{enumerate}

		Debido a la forma en que fue construída de la solución resulta imposible no mencionar la relavancia de contar con herramientas enfocadas a la programación orientada a objetos. 


	
	\section{Trabajo futuro}
	\label{sec:05_trabajo_futuro}
		A corto plazo, es necesario desarrollar e implementar la GUI a modo de completar los objetivos propuestos a este trabajo de título. Para esto, se cuenta con todas las funcionalidades implementadas y lineamientos generales de diseño. Se espera completar este punto en un plazo no superior a un mes desde la entrega de este informe.  

		Como trabajo para futuros desarrollos se proponen las siguientes tareas:
		\begin{enumerate}\setlength\itemsep{-0.2em}
			\item Implementar una mayor variedad de opciones para la configuración de componentes de cuadro, ya sea añadiendo nuevos elementos gráficos como la posibilidad de incluir estímulos sonoros.

			\item Agregar nuevos elementos al script para mejorar y complementar el sistema de registro de eventos en el archivo de salida. Algunas opciones son:
				\begin{enumerate}\setlength\itemsep{-0.2em}
					\item Agregar una lista de los componentes añadidos para cada cuadro con el fin de verificar si fueron cargados correctamente.

					\item Modificar la tabla de registro de eventos de teclado para agregar información sobre si las teclas presionadas corresponden a las requeridas para una tarea específica. Una forma de realizar esto es con un flag que indique si la elección fue correcta. 

					\item Para el caso en que el usuario detiene una sesión durante la ejecución: Incluir una ventana de diálogo para que pueda ser ingresado el motivo de la detención, agregando esta información al registro. 

				\end{enumerate}

			\item Incluir una mayor variedad de preguntas de usuario para el inicio de sesión de un experimento. 

		\end{enumerate}
	

\end{document}
