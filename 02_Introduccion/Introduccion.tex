% =========================================================
% CONFIGURACION DEL DOCUMENTO
% =========================================================
\providecommand{\main}{..}
\documentclass[../main.tex]{subfiles}

% =========================================================
% CONTENIDO
% =========================================================
\begin{document}		
	\chapter[Introducción]{Introducción}
	\pagenumbering{arabic}
	\section[Resumen]{Resumen y Objetivos}	

Una aeronave no tripulada, UAV por sus siglas en inglés, como su nombre lo indica,
es una aeronave sin un piloto humano en su interior,
la cual puede volar de forma autónoma o ser pilotada remotamente,
puede ser desechable o reutilizable y puede transportar cargas letales
o no letales. Los orígenes de las aeronaves no tripuladas se remontan
a principios del siglo XX, donde se utilizaban principalmente como
blancos u objetivos de práctica en entrenamientos militares. Es así
como el desarrollo de este tipo de vehículos creció a la par con la
industria bélica durante la primera y segunda guerra mundial, donde
el principal objetivo era portar explosivos y dirigirlos hacia algún
objetivo específico, como el proyecto Hewitt-Sperry Automatic Airplane,
también conocido como ``flying bomb'' {[}1{]} el cual buscaba desarrollar
un torpedo aéreo.

\textcompwordmark{}

Hace algunas décadas y sin dejar de lado el contínuo desarrollo de
este tipo de vehículos en el ámbito militar, se produjo un cambio
en cuanto a la finalidad con que se construyen. El desarrollo de la
tecnología tanto en la capacidad de procesamiento como en la miniaturización,
permitió la aparición de nuevos usos para los vehículos aéreos no
tripulados, tales como el aeromodelismo, afición que comprende el
estudio de la aerodinámica, mecánica y el diseño de aviones para su construcción
y manejo. 

\textcompwordmark{}

Dentro de los vehículos aereos no tripulados más populares, se encuentra
el denominado quadrotor, haciendo referencia a los cuatro motores
que conforman su propulsión. En los comienzos de la aviación, este
tipo de vehículos fueron los precursores, incluso antes que el helicóptero
convencional, pero dar solución al problema de la estabilidad no era
sencillo debido a que eran controlados manualmente por un operador
que variaba la velocidad de los motores. Tal es el caso del vehículo
``Flying Octopus'' diseñado por George de Bothezad en 1922 {[}2{]},
quadrotor experimental que contó con el auspicio de el Ejercito de
Estados Unidos, el que después arduas pruebas de vuelo decidió abandonar
el proyecto por su alta complejidad y poca fiabilidad.

\textcompwordmark{}

Con los avances tecnológicos en actuadores y sensores de pequeña escala,
tales como los MicroElectoMecanical Systems (MEMS), como así también
en el almacenamiento de energía, capacidad de procesamiento y memoria,
el problema de la estabilidad del quadrotor se pudo resolver.

Es así como hoy en día este tipo de aeronaves son utilizadas en aplicaciones
de agrimensura y agricultura, en seguridad e ingeniería civil en cuanto
a monitorización, industria cinematográfica para la realización de
tomas aereas, operaciones de reconocimiento y rescate, entre otras.

\textcompwordmark{}

Sin embargo, a pesar de las herramientas disponibles, el diseño de
sistemas de control para este tipo de vehículos no es trivial. Los
quadcopter tienen un comportamiento altamente no lineal y es un sistema
que posee más grados de libertad que actuadores. Estas propiedades
hacen que el control de este tipo de vehículos sea un problema interesante
que en algunos casos es resuelto mediante técnicas de control no lineal.

\textcompwordmark{}

Existen dos tipos de problemas de control para quadcopters: estabilización
del vehículo en vuelo hovering y el seguimiento de trayectoria. En
este trabajo se abordará el problema de estabilizar un quadrotor en
vuelo hovering y sobre un soporte que cuenta con los tres ejes de
libertad de giro Pitch, Roll y Yaw. 

Para lograr dicho objetivo, se utilizarán sensores ubicados en la
estructura del vehículo, los que entregarán las medidas de ángulos
y velocidad de giro. Estas serán señales de entrada al controlador
embebido, el que generará las respectivas señales de actuación para
los motores de la planta. Todo lo anterior con el fin de estabilizar
el sistema, robusto a perturbaciones y con seguimiento de referencias
constantes, dado a que también es atractivo el incorporar un control
de los ejes a través de una interfaz HMI.

\textcompwordmark{}

El proyecto comienza con el estudio del modelo no lineal, sub-actuado,
y acoplado de la planta, describiendo las ecuaciones que rigen el
comportamiento físico del vehículo, tomando en cuenta todas las variables
de interés. Una vez modelada la planta es necesario caracterizarla,
por medio de las mediciones correspondientes para describir las características
de los actuadores y las inercias de las partes que conforman la estructura
completa del vehículo. 

Luego, después del análisis del modelo y caracterización, se debe
estudiar de qué forma simplificar el modelo con tal de afrontar el
problema mediante la utilización de herramientas de control lineal.
Una vez hecha la simplificación, se diseña la ley de control que estabilizará
el sistema y posteriormente se simula dicho controlador sobre el modelo
real del vehículo para verificar su funcionamiento. Para la simulación
se utiliza el software MATLAB.

El último objetivo a desarrollar durante este trabajo, es la implementación
del controlador diseñado de forma embebida en la planta real. Para
esto se divide el trabajo en el control de los tres subsistemas por
separado, es decir, se hacen pruebas previas de estabilización y seguimiento
a cada eje de rotación de forma individual, para luego poner a prueba
los controladores sintonizados sobre la planta completa. Para esto
se utiliza el controlador myRIO 1900 de National Instruments y se
programa la adquisición, procesamiento y actuación sobre la planta
con el software LabVIEW versión 2015. Posterior a la implementación
se procede a analizar los resultados obtenidos en los experimentos
en comparación con lo simulado, con el diseño original del controlador
y se estudian las posibles mejoras.

\textcompwordmark{}

\section[Estado del Arte]{Estado del Arte}

Existen hoy en día variados estudios asociados al uso de quadcopters
en diferentes aplicaciones. Se parte por describir algunas aplicaciones
que no son parte del objetivo principal de este proyecto, pero reflejan
el estado actual de la investigación y desarrollo en el control de
éste tipo de vehículos.

Un problema común en aplicaciones que utilizan quadcopters es el de
cargar y descargar objetos variando la masa total del vehículo y por
lo tanto la inercia del sistema. Generalmente los controladores para
la estabilización están sintonizados para controlar el drone sin carga
alguna o para una carga constante, por lo tanto, cambiar la masa significa
un deterioro del control estabilizador. Emran en {[}3{]} presenta
el diseño de una ley de control robusto y adaptivo para este tipo
de aplicaciones de carga dinámica, demostrando la eficiencia del control
mediante el seguimiento de posición y velocidad a través de simulaciones. 

Otro campo de investigación con quadcopters es el de navegación autónoma,
el cual requiere de mecanismos avanzados para la exploración del ambiente
en que se encuentra el vehículo. Bipin en {[}4{]} propone un método
en el que soluciona el problema de percepción mediante el uso de una
cámara monocular y un algoritmo que genera una trayectoria libre de
colisiones por un tiempo mínimo en quadcopters comerciales. El marco
propuesto utiliza en primera instancia herramientas de aprendizaje
supervisado (SVM) para estimar la profundidad del entorno con las
imágenes obtenidas por la cámara frontal. Luego el mapeo de profundidad
es transformado al Ego-Dynamic Space que estima la distancia efectiva
a recorrer en función de los parámetros de desaceleración máxima del
vehículo, posteriormente se procesa esta información utilizando técnicas
de programación entera y convexa para generar una trayectoria libre
de colisiones.

El transporte de carga autónomo es otra área de desarrollo importante
en aplicaciones que requieren el trabajo colaborativo entre drones.
Es así como Mellinger {[}5{]} estudia el problema de controlar mutiples
vehículos que cooperativamente trasladan cargas en tres dimensiones,
modelando el quadrotor tanto individual como colectivamente, sujetos
rígidamente a una carga. Luego se realiza un estudio experimental
con grupos de drones que transportan cargas en tres dimensiones por
trayectorias definidas y en diferentes configuraciones. 

\textcompwordmark{}

Luego de esta breve descripción de algunas de las aplicaciones actuales
en control de quadcopters, se consideran los trabajos relacionados
con los objetivos de este proyecto de titulación. A continuación se
presentan algunos trabajos relacionados con los tópicos a tratar durante
el desarrollo del proyecto.

En {[}6{]} se considera el modelado simplificado de un quadrotor,
es decir, tomando en cuenta un número mínimo de estados y entradas,
sin perder las características fundamentales para el diseño de la
ley de control. La similitud de este trabajo y el que se desea desarrollar
es que en ningún caso es conveniente tomar en cuenta la dinámica completa
del vehículo aéreo, por lo tanto, en ambos se desprecian los efectos
aero-elásticos, flexibilidad de las hélices, dinámicas internas de
los motores, etc, pues el considerarlas dificulta en gran medida el
diseño de una ley de control lineal.

En {[}7{]} se caracterizan los actuadores por medio de mediciones
de empuje de motores y hélices sobre bancos de prueba, con el objetivo
de relacionar características de velocidad angular versus empuje,
las que serán posteriormente utilizadas para simulación e implementación
del control. Una vez obtenidas las características antes mencionadas,
se procede a medir la respuesta del motor en lazo abierto para determinar
su dinámica interna. Con este último experimento, resulta válido suponer
que el actuador responde instantáneamente a la señal de control, o
por lo menos, mucho más rápido que el resto de las dinámicas del vehículo.
En este trabajo se tomará dicho supuesto como válido para simulación
e implementación.

En {[}8{]} se implementan estrategias de control lineal sobre la planta,
realizando las simplificaciones necesarias para aplicar las técnicas
de control por eje mediante controladores PID descentralizados en
comparación con el enfoque de control LQ mediante la representación
de la planta en variables de estado. En este proyecto se implementa
la estrategia de control lineal PID por grado de libertad de giro,
linealizando el modelo de la planta en el punto de operación hovering,
lo que permite despreciar las dinámicas no lineales y acopladas del
sistema.

\textcompwordmark{}


\end{document}
