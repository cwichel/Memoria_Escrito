% ========================================================= Glosario
\newglossaryentry{ejevisual}{
	name={ejes visuales},
	description={
		Corresponde a la proyección de una línea recta que pasa simultáneamente por el centro de la \gls{fovea}, y la pupila.
	}
}

\newglossaryentry{esclerotida}{
	name={esclerótida},
	description={
		Sección blanca del ojo que rodea a la pupila
	}
}

\newglossaryentry{eyetracker}{
	name={eye tracker},
	description={
		Dispositivo utilizado para realizar seguimiento de los movimientos oculares
	}
}

\newglossaryentry{fovea}{
	name={fóvea},
	description={
		Área de la retina que permite la visión más nítida y detallada
	}
}

\newglossaryentry{setup}{
	name={setup},
	description={
		Se asigna esta denominación a cierta configuración de un espacio de trabajo
	}
}

% ========================================================= Acronimos
\newacronym{crt}{CRT}{Cathode Ray Tube (Tubo de Rayos Catódicos)}
\newacronym{eog}{EOG}{Electro-OculoGraphy (Electro-oclulografía)}
\newacronym{fp}{FP}{Fixation Point (Punto de Fijación)}
\newacronym{fps}{FPS}{Frames Per Second (Cuadros Por Segundo)}
\newacronym{gui}{GUI}{Graphical User Interface (Interfaz gráfica de Usuario)}
\newacronym{isi}{ISI}{Inter Saccadic Inteval (Intervalo Inter-Sacádico)}
\newacronym{rt}{RT}{Response Time (Tiempo de Respuesta)}
\newacronym{sl}{SL}{Saccadic Latency (Latencia Sacádica)}
\newacronym{ssc}{SSC}{Scleral Search Coil (Bobina Escleral de Búsqueda)}
\newacronym{tp}{TP}{Target Point (Punto Objetivo)}
\newacronym{vog}{VOG}{Video-OculoGraphy (Oculografía basada en Video) }
