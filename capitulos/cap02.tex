% =========================================================
% CONFIGURACION DEL DOCUMENTO
% =========================================================
\providecommand{\main}{..}
\documentclass[../main.tex]{subfiles}

% =========================================================
% CONTENIDO
% =========================================================
\begin{document}
\chapter{Modelado de la Planta}\label{Modelado de la Planta}

En el desarrollo de este capítulo, se explican los conceptos utilizados
para realizar el modelo de la planta en vuelo libre. Un Quadrotor es controlado mediante
la actuación de cuatro motores eléctricos como se muestra en la figura
\ref{fig:Diagrama-de-Cuerpo }. Cada motor con su respectiva hélice
acoplada produce empuje y torque, los cuales al combinarse generan
el empuje principal y los momentos yaw, pitch y roll. Cabe destacar
desde ya que el sistema es subactuado en vuelo libre. Dado el vector de entradas
a la planta (el torque y empuje generado por los cuatro motores),
éste no puede producir un movimiento del vehículo en una trayectoria
arbitraria, sus entradas son menos que sus salidas, en otras palabras,
el vehículo no puede moverse horizontalmente sin tener que cambiar
su inclinación y altura. 

De la figura \ref{fig:Diagrama-de-Cuerpo } se puede observar que
los motores $M_{i}$ para ($i=1,...,4$) producen las fuerzas $f_{i}$
y los torques $\tau_{M_{i}}$ y el empuje principal $u$, los que serán
descritos posteriormente. Los motores del frente $M_{1}$ y de la
parte posterior $M_{3}$ giran en contra del sentido de las manecillas
del reloj, mientras que los motores de izquierda y derecha, $M_{2}$ y
$M_{4}$ respectivamente, giran en el sentido de las manecillas del
reloj. Con ésto, los efectos giroscópicos y torques aerodinámicos
se tienden a cancelar.

\begin{figure}[H]
\noindent \begin{centering}
\includegraphics[scale=0.6]{\string"Quadrotor Modelado 01\string".jpg}
\par\end{centering}
\caption{\label{fig:Diagrama-de-Cuerpo }Diagrama de Cuerpo Libre de Quadrotor.}
\end{figure}

\textcompwordmark{}

Para un mejor entendimiento, se separa el capítulo de modelado en
cuatro secciones que cubren el comportamiento aerodinámico del vehículo,
la relación entre los marcos de referencia, el enfoque lagrangeano y el enfoque de Newton-Euler estudiados para el modelado.


\subsection{Aerodinámica de Motores: Momentos y Hélices}

Para desarrollar un modelo del comportamiento de la aerodinámica de
las hélices y motores, se considera que el empuje esta dado por el
cambio en el momentum del aire que es desplazado hacia abajo por el
giro de la hélice. Bajo esta lógica, la relación entre la magnitud
del empuje y la magnitud de la velocidad inducida en el aire [9] es: 

\begin{equation}
f_{i}=2\rho Av_{i}^{2}
\end{equation}

Donde $f_{i}$ es el empuje iésimo, $\rho$ es la densidad del aire,
$A$ es el área del disco que cubre la hélice al girar y $v_{i}$
es la velocidad inducida al aire por el motor iésimo. 
El empuje total también puede ser expresado en función de la velocidad
angular del rotor [10]  mediante la siguiente ecuación:

\begin{equation}
f_{i}=C_{T}\rho A(\omega R)^{2}
\end{equation}

donde $f_{i}$ es el empuje total producido por el rotor iésimo, $R$
es el radio del rotor y $C_{T}$ es un coeficiente adimensional de
empuje. Por lo tanto, se tienen dos formas de levantar un cuerpo por
la acción de una hélice giratoria. Se puede tener una hélice de diametro
pequeño girando a gran velocidad, o se puede tener una hélice de diámetro
mayor girando a menor velocidad.

Para la implementación del proyecto se cuenta con hélices que poseen
las mismas características, el área cubierta por cada una es igual.
Por otro lado la densidad del aire es contante y el radio del rotor
también, lo que permite concentrar todas estas constantes en una sola
y reescribir la ecuación simplificada.

\begin{equation}
f(\omega_{i})=b\cdot\omega_{i}^{2}
\end{equation}

En general la constate $b$ es dificil de estimar, dado que no se
conocen los parámetros aerodinámicos de las hélices con exactitud,
puesto que las especificaciones del fabricante no están disponibles.
Por lo tanto, deberá ser determinada empíricamente por medio de un
banco de pruebas que será descrito posteriormente.

De manera similar, el torque de cada motor para maniobras casi estacionarias
se puede representar mediante una relación cuadratica de la velocidad
angular [11].

\begin{equation}
\tau_{M_{i}}(\omega_{i})=K_{M}\omega_{i}^{2}
\end{equation}

Para determinar la constante $K_{M}$ también es necesario realizar
pruebas para determinarla.
Entonces como se muestra en la figura \ref{fig:Diagrama-de-Cuerpo },
la suma de cada una de las fuerzas de empuje desplazadas hacia el
centro de masa del vehículo corresponde al empuje principal $u.$
Describiendolo matemáticamente se tiene:

\begin{equation}
u=\sum_{i=1}^{4}f_{i}\label{eq:Empuje Total}
\end{equation}

El movimiento de giro pitch hacia adelante se obtiene disminuyendo
la velocidad de giro del motor $M_{1}$ y aumentando la velocidad
del motor $M_{3}$. Así también para el moviento roll hacia la derecha,
basta con disminuir la velocidad del motor $M_{4}$ y aumentar la
velocidad del motor $M_{2}$. Por último el movimiento en yaw se logra
incrementando el torque (por lo tanto también velocidad angular) de
los motores delantero $M_{1}$ y trasero $M_{3}$, mientras se disminuye
el torque de los motores laterales $M_{2}$ y $M_{4}$. Los tipos
de movimientos descritos se visualizan de mejor forma en la siguiente
figura.

\begin{figure}[H]
\noindent \begin{centering}
\includegraphics[scale=0.6]{\string"Quadrotor Modelado 02\string".jpg}
\par\end{centering}
\caption{Torques yaw, pitch y roll de vehículo Quadrotor.}
\end{figure}

\textcompwordmark{}

Escribiendo matemáticamente los momentos resultantes de la interacción
de los torques y fuerzas producidos por los motores descritos en el
párrafo anterior, se tiene la siguiente equivalencia:

\begin{equation}
\mathbf{\tau}=\left[\begin{array}{c}
\tau_{\phi}\\
\tau_{\theta}\\
\tau_{\psi}
\end{array}\right]\triangleq\begin{bmatrix}(f_{2}-f_{4})\cdot\ell\\
(f_{3}-f_{1})\cdot\ell\\
\tau_{M_{1}}+\tau_{M_{3}}-\tau_{M_{2}}-\tau_{M_{4}}
\end{bmatrix}\label{eq:Momentos Rotacionales}
\end{equation}

Donde $\phi$ representa al ángulo en el eje de alabeo Roll, $\theta$ en el eje de elevación Pitch y $\psi$ el giro en el ángulo de dirección Yaw. Además, $\ell$ representa la distancia desde el rotor hasta el centro
de masa del vehículo y $\tau_{M_{i}}$ para $i=1,...,4$ son los torques
producidos por lo motores.

\subsection{Marcos de Referencia y Matriz de Rotación}

Para el análisis de la dinámica del quadrotor existen dos marcos de
coordenadas, el primero es el marco de referencia cuyo origen está
en el centro de masa del vehículo, lo denominaremos $\mathcal{A}$,
mientras que el segundo corresponde al sistema de referencia inercial
en tierra que lo denominaremos $\mathcal{I}$. La siguiente figura
muestra dichos sistemas de referencia. 

\noindent \begin{center}
\begin{figure}[H]
\noindent \begin{centering}
\includegraphics[scale=0.6]{\string"Quadrotor Modelado 03\string".jpg}
\par\end{centering}
\caption{Sistemas de coordenadas para análisis dinámico del quadrotor.}
\end{figure}
\par\end{center}

Dado que los sensores estarán fijos en la estructura del vehículo,
las mediciones se corresponden con el sistema de referencia $\mathcal{A}$
que rota en conjunto con éste. Por lo tanto, si el objetivo fuese
hacer un control de trayectoria, se necesita adecuar estas mediciones
al sistema de referencia inercial en tierra. Si bien el objetivo final
no es el mencionado anteriormente, se modelarán y simularán todos
los aspectos del vuelo del vehículo, tanto los rotacionales como los
traslacionales. Por lo tanto para obtener las respectivas aceleraciones
y velocidades en sistema inercial $I$, es necesaria la inclusión
de una matriz de rotación $R$, que se deduce a continuación con la
ayuda de la figura \ref{fig:Giro-de-sistema}, donde se muestra la
rotación del sistema de referencia en los ángulos pitch, roll y
yaw.

\begin{figure}[H]
\noindent \begin{centering}
\includegraphics[scale=0.6]{\string"Quadrotor Modelado 05\string".png}
\par\end{centering}
\caption{\label{fig:Giro-de-sistema}Giro de sistema de referencia $\mathcal{A}$
respecto a sistema inercial $\mathcal{I}$ compartiendo el origen.}
\end{figure}


Supongamos que se hace una rotación en el plano $x-y$, dejando el
eje $z$ fijo y utilizándolo como eje de rotación. De acuerdo al supuesto,
el sistema de referencia $\mathcal{A}$ está girando respecto al sistema
inercial $\mathcal{I}$ una cantidad definida por el ángulo de rotación
yaw $\psi$. Dicha rotación tiene asociada una matriz de transformación, por lo tanto se descomponen los nuevos
ejes en los vectores que los conforman. Utilizando la notación que
describe la figura \ref{fig:Giro-de-sistema} se tiene la siguiente
equivalencia:

\begin{equation}
\begin{pmatrix}x\\
y\\
z
\end{pmatrix}=[T_{z}]\cdot\begin{pmatrix}x'\\
y'\\
z'
\end{pmatrix}
\end{equation}

Donde $T_{z}$ representa la matriz de rotación en torno al eje $z$. 

\begin{equation}
[T_{z}]=\begin{pmatrix}cos\psi & sin\psi & 0\\
-sin\psi & cos\psi & 0\\
0 & 0 & 1
\end{pmatrix}
\end{equation}

Siguiendo la misma lógica, los giros en dirección del ángulo pitch
$\theta$ y roll $\phi$ quedan descritos por las matrices $T_{y}$
y $T_{x}$ respectivamente.

\begin{equation}
[T_{x}]=\begin{pmatrix}1 & 0 & 0\\
0 & cos\phi & sin\phi\\
0 & -sin\phi & cos\phi
\end{pmatrix}\qquad\qquad[T_{y}]=\begin{pmatrix}cos\theta & 0 & -sin\theta\\
0 & 1 & 0\\
sin\theta & 0 & cos\theta
\end{pmatrix}
\end{equation}

Así se obtiene la matrizde rotación $R$ del sistema en los tres ejes,
donde $c\theta$ representa $cos\theta$y $s\theta$ representa $sin\theta$.

\begin{equation}
\begin{pmatrix}x\\
y\\
z
\end{pmatrix}=[T_{x}][T_{y}][T_{z}]\begin{pmatrix}x'\\
y'\\
z'
\end{pmatrix}=\begin{pmatrix}c\theta c\psi & c\theta s\psi & -s\theta\\
s\phi s\theta c\psi-c\phi s\psi & s\phi s\theta s\psi+c\phi c\psi & s\phi c\theta\\
c\phi s\theta c\psi+s\phi s\psi & c\phi s\theta s\psi-s\phi c\psi & c\phi c\theta
\end{pmatrix}\begin{pmatrix}x'\\
y'\\
z'
\end{pmatrix}
\end{equation}

Por otro lado para saber como influye la razón de cambio de los ángulos
de Euler en el vector velocidad angular en el marco $\mathcal{A}$,
que apunta en dirección al eje de rotación, se tiene la siguiente
definición [12]:

\begin{equation}
\omega=\begin{pmatrix}\omega_{x}\\
\omega_{y}\\
\omega_{z}
\end{pmatrix}=Id\cdot\begin{pmatrix}\dot{\phi}\\
0\\
0
\end{pmatrix}+[T_{x}]\cdot\begin{pmatrix}0\\
\dot{\theta}\\
0
\end{pmatrix}+[T_{x}][T_{y}]\cdot\begin{pmatrix}0\\
0\\
\dot{\psi}
\end{pmatrix}
\end{equation}

\begin{equation}
\omega=\begin{bmatrix}1 & 0 & -sin\theta\\
0 & cos\phi & cos\theta sin\phi\\
0 & -sin\phi & cos\theta cos\phi
\end{bmatrix}\begin{bmatrix}\dot{\phi}\\
\dot{\theta}\\
\dot{\psi}
\end{bmatrix}=W_{n}\dot{\eta}
\end{equation}

\textcompwordmark{}

\subsection{Modelo dinámico del quadrotor: Enfoque de Euler Lagrange}

En el libro Quad Rotorcraft Control Vision-Based Hovering and Navigation,
se utiliza este método para modelar la dinámica del quadrotor. El
modelo se sustenta en el supuesto de que las dinámicas de los motores
son mucho más rápidas que las del sistema, por lo tanto pueden ser
despreciadas. Por otro lado, las hélices de los motores se asumen
simétricas, iguales y rígidas. Además se asume un modelo donde se
desprecian los efectos de perturbaciones causadas por el viento en
ambientes exteriores.

Las ecuaciones de movimiento del vehículo se desarrollan a partir
de las expresiones de energía cinética traslacional y rotacional del
sistema. Sea el vector $q$ de coordenadas
generalizadas referidas al sistema inercial en tierra $\mathcal{I}$:

\begin{equation}
q=[x\ y\ z\ \phi\ \theta\ \psi]^{T}
\end{equation}
 
Esta coordenadas generales pueden ser separadas en componentes traslacionales
y rotacionales

\begin{equation}
\xi=[x\ y\ z]^{T}\quad y\quad\eta=[\phi\ \theta\ \psi]^{T}
\end{equation}

Por lo tanto se tiene 

\begin{equation}
q=[\xi\quad\eta]^{T}
\end{equation}

El Lagrangeano se obtiene mediante la diferencia entre la energía
cinética y energía potencial del sistema. La energía cinética tiene
componentes rotacionales y traslacionales, mientras que la energía
potencial está relacionada con la altura del centro de masa del vehículo.
Por lo tanto el Lagrangeano del vehículo se puede expresar como:

\begin{equation}
L=E_{c}-E_{p}
\end{equation}

Donde $E_{c}$ es la energía cinética del sistema y $E_{p}$ corresponde
a la energía potencial. La energía cinética total del sistema es la
suma de la energía del movimiento traslacional y del rotacional, mientras
que la potencial depende de la altura y la masa constante del vehículo. 

Antes de enunciar las expresiones para dichas energías, se define
la matriz $\mathbb{J}=\mathbb{J}(\eta)=W_{\eta}^{T}IW_{\eta}$, que
representa la matriz de inercia para la cinética rotacional expresada
en términos de las coordenadas generalizadas $\eta$.

Por lo tanto la expresión para el Lagrangeano es finalmente:

\begin{equation}
L(q,\dot{q})=\frac{1}{2}\dot{\xi}^{T}m\dot{\xi}+\frac{1}{2}\dot{\eta}^{T}\mathbb{J}\dot{\eta}-mgz
\end{equation}

Dado que el Lagrangeano no contiene términos cruzados entre la energía
cinética traslacional y la energía cinética rotacional, la ecuación
de Euler-Lagrange puede ser dividida en dinámicas diferentes para
cada tipo de movimiento, para las coordenadas $\xi$ y $\eta$.

La ecuación de Euler-Lagrange para el movimiento traslacional es:

\begin{equation}
\frac{d}{dt}\left[\frac{\partial L_{Tras}}{\partial\dot{\xi}}\right]-\frac{\partial L_{Tras}}{\partial\xi}=F_{\xi}
\end{equation}


Por lo tanto la dinámica traslacional es finalmente:

\begin{equation}
m\ddot{\xi}+mg\hat{z}=F_{\xi}
\end{equation}

Las fuerzas traslacionales son las de empuje resultantes de cada motor,
y las fuerzas rotacionales son las resultantes de los momentos causados
por las fuerzas de los motores opuestos sobre el centro de gravedad.
Dado que para el desarrollo del modelo no se están considerando los
efectos giroscópicos y se mantiene lo más simple posible, la fuerza
traslacional está dada por:

\begin{equation}
F_{rotor}=[0\ 0\ u]^{T}
\end{equation}

Donde $u$ es el empuje total como se describe en la ecuación \ref{eq:Empuje Total}.
Es claro que dicha fuerza de empuje apunta en la dirección del eje
z, pero dado que los sensores adquieren los datos para el sistema
de referencia que se mueve junto al vehículo, se debe relacionar esta
cantidad respecto al sistema inercial descrito en la sección anterior.
Por lo tanto, 

\begin{equation}
F_{\xi}=R\cdot F_{rotor}=\begin{bmatrix}-u\cdot(s\phi s\psi+c\phi c\psi s\theta)\\
u\cdot(c\phi s\theta s\psi-c\psi s\phi)\\
u\cdot c\phi c\theta
\end{bmatrix}
\end{equation}

Escribiendo la dinámica traslacional de forma matricial se tiene:

\begin{equation}
\begin{bmatrix}\ddot{x}\\
\ddot{y}\\
\ddot{z}
\end{bmatrix}=-g\cdot\begin{bmatrix}0\\
0\\
1
\end{bmatrix}+\frac{u}{m}\cdot\begin{bmatrix}-(s\phi s\psi+c\phi c\psi s\theta)\\
c\phi s\theta s\psi-c\psi s\phi\\
c\phi c\theta
\end{bmatrix}
\end{equation}

Luego la ecuación de Euler-Lagrange para el movimiento rotacional
es:

\begin{equation}
\frac{d}{dt}\left[\frac{\partial L_{Rot}}{\partial\dot{\eta}}\right]-\frac{\partial L_{Rot}}{\partial\eta}=\tau
\end{equation}

\begin{equation}
\frac{d}{dt}\left[\frac{1}{2}\frac{\partial(\dot{\eta}^{T}\mathbb{J}\dot{\eta})}{\partial\dot{\eta}}\right]-\frac{1}{2}\frac{\partial(\dot{\eta}^{T}\mathbb{J}\dot{\eta})}{\partial\eta}=\tau
\end{equation}

\begin{equation}
\mathbb{J}\ddot{\eta}+\dot{\mathbb{J}}\dot{\eta}-\frac{1}{2}\frac{\partial}{\partial\eta}(\dot{\eta}^{T}\mathbb{J}\dot{\eta})=\tau
\end{equation}

\begin{equation}
\mathbb{J}\ddot{\eta}+C(\eta,\dot{\eta})\dot{\eta}=\tau\label{eq:rotacional lagrangeano}
\end{equation}

El término $C(\eta,\dot{\eta})$ de la ecuación (\ref{eq:rotacional lagrangeano})
es conocido como matriz de Coriolis, que contiene las componentes
giroscópicas y centrípetas. El modelo anterior es de segundo orden
y no lineal, dadas las matrices $\mathbb{J}\ y\ C(\eta,\dot{\eta})$.

\begin{equation}
\mathbb{J}=\begin{bmatrix}I_{x} & 0 & -S_{\theta}I_{x}\\
0 & I_{y}C_{\phi}^{2}-I_{z}S_{\phi}^{2} & (I_{y}-I_{z})C_{\phi}C_{\theta}S_{\phi}\\
-S_{\theta}I_{x} & (I_{y}-I_{z})C_{\phi}C_{\theta}S_{\phi} & S_{\theta}^{2}I_{x}+I_{y}C_{\theta}^{2}S_{\phi}^{2}+I_{z}C_{\theta}^{2}C_{\phi}^{2}
\end{bmatrix}
\end{equation}

\begin{equation}
C(\eta,\dot{\eta})=(\dot{\mathbb{J}}-\frac{1}{2}\frac{\partial}{\partial\eta}(\dot{\eta}^{T}\mathbb{J}))=\begin{bmatrix}C_{11} & C_{12} & C_{13}\\
C_{21} & C_{22} & C_{23}\\
C_{31} & C_{32} & C_{33}
\end{bmatrix}
\end{equation}

Donde:
\begin{equation}
C_{11}=0;\quad C_{12}=(I_{y}-I_{z})(\dot{\theta}C_{\phi}S_{\phi}+\dot{\psi}S_{\phi}^{2}C_{\theta})+(I_{z}-I_{y})\dot{\psi}C_{\phi}^{2}C_{\theta}-I_{x}\dot{\psi}C_{\theta}
\end{equation}

\begin{equation}
C_{13}=(I_{z}-I_{y})\dot{\psi}C_{\phi}S_{\phi}C_{\theta}^{2}
\end{equation}

\begin{equation}
C_{21}=(I_{z}-I_{y})\cdot(\dot{\theta}C_{\phi}S_{\phi}+\dot{\psi}S_{\phi}C_{\theta})+(I_{y}-I_{z})\dot{\psi}C_{\phi}^{2}C_{\theta}+I_{x}\dot{\psi}C_{\theta}
\end{equation}

\begin{equation}
C_{22}=(I_{z}-I_{y})\dot{\phi}C_{\phi}S_{\phi}
\end{equation}

\begin{equation}
C_{23}=-I_{x}\dot{\psi}S_{\theta}C_{\theta}+I_{y}\dot{\psi}S_{\phi}^{2}S_{\theta}C_{\theta}+I_{z}\dot{\psi}C_{\phi}^{2}S_{\theta}C_{\theta}
\end{equation}

\begin{equation}
C_{31}=(I_{y}-I_{z})\dot{\psi}C_{\theta}^{2}S_{\phi}C_{\theta}-I_{x}\dot{\theta}C_{\theta}
\end{equation}

\begin{equation}
C_{32}=(I_{z}-I_{y})\cdot(\dot{\theta}C_{\phi}S_{\phi}S_{\theta}+\dot{\phi}S_{\phi}^{2}C_{\theta})+(I_{y}-I_{z})\dot{\phi}C_{\phi}^{2}C_{\theta}+I_{x}\dot{\psi}S_{\theta}C_{\theta}-I_{y}\dot{\psi}S_{\phi}^{2}S_{\theta}C_{\theta}-I_{z}\dot{\psi}C_{\phi}^{2}S_{\theta}C_{\theta}
\end{equation}

\begin{equation}
C_{33}=(I_{y}-I_{z})\dot{\phi}C_{\phi}S_{\theta}C_{\theta}^{2}-I_{y}\dot{\theta}S_{\phi}^{2}C_{\theta}S_{\theta}-I_{z}\dot{\theta}C_{\phi}^{2}C_{\theta}S_{\theta}+I_{x}\dot{\theta}C_{\theta}S_{\theta}
\end{equation}

La ecuación (\ref{eq:rotacional lagrangeano}), conduce a las ecuaciones
diferenciales para el modelo de la aceleración angular.

\begin{equation}
\ddot{\eta}=\mathbb{J}^{-1}(\tau-C(\eta,\dot{\eta})\dot{\eta})
\end{equation}

\subsection{Modelo dinámico del quadrotor: Enfoque de Newton Euler}

Dada la extensión y complejidad de la parte rotacional del modelo
previamente deducido mediante el enfoque de Euler-Lagrange, es que
se considera desarrollar un segundo enfoque para el modelado de dicha
dinámica. Ésto surge como estrategia para la etapa de simulación,
donde se utiliza el modelo equivalente de Newton-Euler, el cual es
más simple de implementar. 

En primer lugar se supondrá que el vehículo es rígido, por lo cual
se puede utilizar la ecuación de dinámica de cuerpos rígidos en el
sistema de referencia $\mathcal{A}$ (cuerpo del vehículo). También
se asume que el vehículo posee una estructura simétrica con los cuatro
brazo alineados con los ejes $x$ e $y$, por lo tanto su matriz de
inercia es constante, diagonal y simétrica con $I_{x}=I_{y}$.

Antes del desarrollo de la ecuación, se define el vector de velocidad
angular en el sistema de referencia del vehículo.

\begin{equation}
\nu=W_{\eta}\dot{\eta}\ \implies\ \nu=\begin{bmatrix}p\\
q\\
r
\end{bmatrix}=\begin{bmatrix}1 & 0 & -S_{\theta}\\
0 & C_{\phi} & C_{\theta}S_{\phi}\\
0 & -S_{\phi} & C_{\theta}C_{\phi}
\end{bmatrix}\begin{bmatrix}\dot{\phi}\\
\dot{\theta}\\
\dot{\psi}
\end{bmatrix}\label{eq:138}
\end{equation}

Donde la matriz $W_{\eta}$ será invertible si $\theta\neq(2k-1)\phi/2,\ k\in\mathbb{Z}.$ 

Por lo tanto la ecuación de Euler para la dinámica de cuerpos rígidos
es:

\begin{equation}
I\dot{v}+\nu\times(I\nu)=\tau
\end{equation}

\begin{equation}
\dot{\nu}=I^{-1}\left(-\begin{bmatrix}p\\
q\\
r
\end{bmatrix}\times\begin{bmatrix}I_{x}p\\
I_{y}q\\
I_{z}r
\end{bmatrix}+\tau\right)
\end{equation}

\begin{equation}
\begin{bmatrix}\dot{p}\\
\dot{q}\\
\dot{r}
\end{bmatrix}=\begin{bmatrix}(I_{y}-I_{z})qr/I_{x}\\
(I_{z}-I_{x})pr/I_{y}\\
(I_{x}-I_{y})pq/I_{z}
\end{bmatrix}+\begin{bmatrix}\tau_{\phi}/I_{x}\\
\tau_{\theta}/I_{y}\\
\tau_{\psi}/I_{z}
\end{bmatrix}\label{eq:141}
\end{equation}

Por lo tanto este modelo describe el vector aceleración en el marco
inercial del vehículo, la que es más simple de implementar en la simulación.
Para obtener el modelo equivalente deducido por el método de Euler-Lagrange,
se puede obtener la aceleración angular en el marco inercial mediante
la matriz $W_{\eta}^{-1}$ y su derivada en el tiempo 

\begin{equation}
\ddot{\eta}=\frac{d}{dt}(W_{\eta}^{-1}\nu)=\frac{d}{dt}(W_{\eta}^{-1})\nu+W_{\eta}\dot{\nu}
\end{equation}

\textcompwordmark{}

\end{document}