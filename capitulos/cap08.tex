% =========================================================
% CONFIGURACION DEL DOCUMENTO
% =========================================================
\providecommand{\main}{..}
\documentclass[../main.tex]{subfiles}

% =========================================================
% CONTENIDO
% =========================================================
\begin{document}

\chapter{Conclusiones y Trabajos Futuros}

En este capítulo se exponen las conclusiones del proyecto realizado, los objetivos alcanzados, posibles mejoras a la implementación y trabajos futuros de interés.

\section{Conclusiones}

Para respetar el orden en que se desarrolló el proyecto, en primer lugar se puede concluir que el modelo linealizado en torno al punto de operación hovering y la estrategia de control lineal utilizada, funciona de forma completa en simulación, sin embargo, no funciona en la etapa de implementación. Lo anterior se debe a distintos factores. Uno de ellos es el soporte sobre el cual se testea la estrategia, el cual presenta gran resistencia al movimiento y cambia la dinámica de la planta. Como así también, otra causa es la implementación misma del control, con bucles de labview que no aseguran una ejecución en un tiempo determinado. Otra razón por la cual el control no estabiliza al sistema es por el ruido en la medición de la posición debido al funcionamiento de los motores, el cual no fue considerado en la etapa de simulación.

\hfill \break

Luego de realizar los experimentos con la primera estrategia en el soporte de metal, se implementó la segunda estrategia de control sobre un soporte construido con un coeficiente de roce menor y con mayor similitud a la dinámica del vehículo en vuelo hovering. Los resultados con el control PID en cascada fueron satisfactorios para la estabilización de cada eje por separado, sin embargo, al testear la solución con el vehículo libre (sin soporte en vuelo), tampoco se obtienen buenos resultados. Lo anterior se debe principalmente a que no se cuenta con el espacio necesario para probar la partida del vehículo con desplazamientos laterales mayores a 2[m], dado que las cuerdas de seguridad con que se sujeta el vehículo interfieren desestabilizándolo desde la partida en adelante.

\hfill \break

\section{Trabajos Futuros}
Dado el alcance de este proyecto y los objetivos alcanzados en su desarrollo, surgen varias posibles alternativas de trabajos futuros a realizar.
\hfill \break
En primer lugar, dado que en este proyecto se alcanza con éxito el estudio del modelo y la simulación del sistema, pero no la implementación de forma completa, se sugiere como trabajo futuro un rediseño de la plataforma de pruebas o soporte de tres grados de libertad. Esto con el fin de no modificar o alterar la dinámica de la planta, para así testear de mejor forma y más segura, el desempeño de estrategias de control sobre la planta.
\hfill \break
Además de modificar la estructura del soporte, conviene incluir nuevos sensores en éste, como por ejemplo encoders que puedan medir el giro de los ejes del soporte, para obtener todos los ángulos de inclinación sin la necesidad de implementar integración numérica sobre sensores digitales (gyroscopo).
\hfill \break
Por otra parte, dado que la dinámica del vehículo es multivariable, acoplada, subactuada y no lineal, se recomienda el estudio de herramientas de control no lineal, junto con el estudio e implementación de filtros activos como Filtro de Kalmann [20] y Filtro de Wiener [21], para el manejo del ruido y la estimación de ángulos. Este sería un avance en la obtención de la posición del vehículo, agregando mayor precisión en dicha medida.
\hfill \break
Adicional a la estabilización del vehículo sobre el soporte descrito y probar nuevas estrategias de control, se puede incluir en la HMI la adquisición de video en tiempo real de la planta, junto con los gráficos de posición y actuación del vehpiculo, con el fin de poder visualizar de mejor forma el funcionamiento del control. 
\hfill \break
Por último, una vez realizada la estabilización en el soporte, se pueden realizar pruebas de control en vuelo hovering, con el fin de que el siguiente desarrollo contemple medición de posición en el espacio para el seguimiento de trayectorias.

\end{document}
