% =========================================================
% CONFIGURACION DEL DOCUMENTO
% =========================================================
\providecommand{\main}{..}
\documentclass[../main.tex]{subfiles}

% =========================================================
% CONTENIDO
% =========================================================
\begin{document}

\lstset{language=Matlab,%
    %basicstyle=\color{red},
    breaklines=true,%
    morekeywords={matlab2tikz},
    keywordstyle=\color{blue},%
    morekeywords=[2]{1}, keywordstyle=[2]{\color{black}},
    identifierstyle=\color{black},%
    stringstyle=\color{mylilas},
    commentstyle=\color{mygreen},%
    showstringspaces=false,%without this there will be a symbol in the places where there is a space
    numbers=left,%
    numberstyle={\tiny \color{black}},% size of the numbers
    numbersep=9pt, % this defines how far the numbers are from the text
    emph=[1]{for,end,break},emphstyle=[1]\color{blue}, %some words to emphasise
    %emph=[2]{word1,word2}, emphstyle=[2]{style},    
}

\chapter{Anexos del Proyecto}
\section*{Código Matlab Para simulación}

\lstinputlisting{TEST.m}
\lstinputlisting{simulacion.m}
\lstinputlisting{rotation.m}
\lstinputlisting{Referencias.m}

\newpage

\section*{Código LabVIEW para Implementación}
\begin{landscape}

\begin{figure}
\begin{centering}
\includegraphics[scale=0.6]{3D_Picture}
\par\end{centering}
\caption{Visualización de medición de ángulos en 3D. Roll, Pitch y Yaw seteados
como variables compartidas.}
\end{figure}

\begin{figure}
\begin{centering}
\includegraphics[scale=0.6]{3D_PicturePanel}
\par\end{centering}
\caption{Panel frontal simulación 3D para la medición de ángulos.}
\end{figure}

\begin{figure}
\begin{centering}
\includegraphics[scale=0.8]{AngPitchRoll}
\par\end{centering}
\caption{Calculo de ángulos Pitch y Roll con filtro complementario.}
\end{figure}

\begin{figure}
\begin{centering}
\includegraphics[scale=0.8]{AngPitchRollPanel}
\par\end{centering}
\caption{Panel frontal para visualización de ángulos Pitch y Roll con filtro
complementario.}
\end{figure}

\begin{figure}
\begin{centering}
\includegraphics[scale=0.8]{ComFilter}\includegraphics[scale=0.6]{ComFilterPanel}
\par\end{centering}
\caption{Filtro complementario para Pitch y Roll. Une medida de acelerómetro
y gyróscopo a través de ponderación.}
\end{figure}

\begin{figure}
\begin{raggedright}
\includegraphics[scale=0.8]{ConfigAcelOffset}
\par\end{raggedright}
\caption{Calculo de offset para acelerómetro.}
\end{figure}

\begin{figure}
\begin{centering}
\includegraphics[scale=0.8]{ConfigGyroI2C}
\par\end{centering}
\caption{Configuración de Gyróscopo y cálculo de offset.}
\end{figure}

\begin{figure}
\begin{centering}
\includegraphics[scale=0.6]{GyroRead}
\par\end{centering}
\caption{}
\end{figure}

\begin{figure}
\begin{centering}
\includegraphics[scale=0.8]{ConvertGxS}
\par\end{centering}
\caption{Conversión de medida de gyróscopo a grados por segundo.}
\end{figure}

\begin{figure}
\begin{centering}
\includegraphics[scale=0.8]{Yaw__integral}
\par\end{centering}
\caption{Cálculo de integral de velocidad en eje Z para ángulo Yaw.}
\end{figure}

\begin{figure}
\begin{centering}
\includegraphics[scale=0.8]{ThrottleToDuty}
\par\end{centering}
\caption{transformación de acelerador a ciclo de trabajo para calibración de motores.}
\end{figure}

\begin{figure}
\begin{centering}
\includegraphics[scale=0.7]{Calibre}
\par\end{centering}
\caption{Calibre automático de motores.}
\end{figure}

\begin{figure}
\begin{centering}
\includegraphics[scale=0.6]{MainLoop1}
\par\end{centering}
\caption{Loop principal parte 1.}
\end{figure}

\begin{figure}
\begin{centering}
\includegraphics[scale=0.6]{MainLoop2}
\par\end{centering}
\caption{Loop principal parte 2. }
\end{figure}

\begin{figure}
\begin{centering}
\includegraphics[scale=0.8]{PanelFrontal1}
\par\end{centering}
\caption{Panel frontal principal parte 1.}
\end{figure}

\begin{figure}
\begin{centering}
\includegraphics[scale=0.6]{PanelFrontal2}
\par\end{centering}
\caption{Panel frontal principal parte 2.}
\end{figure}

\end{landscape}

\section*{Datos Caracterización de Motores}

\begin{figure}
\begin{centering}
\includegraphics[scale=0.8]{motor1real}
\par\end{centering}
\caption{Mediciones de empuje y velocidad angular para motor 1.}
\end{figure}

\begin{figure}
\begin{centering}
\includegraphics[scale=0.8]{motor2real}
\par\end{centering}
\caption{Mediciones de empuje y velocidad angular para motor 2.}
\end{figure}

\begin{figure}
\begin{centering}
\includegraphics[scale=0.8]{motor3real}
\par\end{centering}
\caption{Mediciones de empuje y velocidad angular para motor 3.}
\end{figure}

\begin{figure}
\begin{centering}
\includegraphics[scale=0.8]{motor4real}
\par\end{centering}
\caption{Mediciones de empuje y velocidad angular para motor 4.}
\end{figure}

\begin{figure}
\begin{centering}
\includegraphics[scale=0.8]{torque}
\par\end{centering}
\caption{Torque v/s velocidad angular.}
\end{figure}

\end{document}