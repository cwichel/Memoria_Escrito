% =========================================================
% CONFIGURACION DEL DOCUMENTO
% =========================================================
\providecommand{\main}{..}
\documentclass[../main.tex]{subfiles}

% =========================================================
% CONTENIDO
% =========================================================
\begin{document}		
	\chapter[Referencias]{Referencias}
	
[1]  http://www.designation-systems.net/dusrm/app4/sperry-fb.html, 
       Directory of U.S. Military Rockets and Missiles, Flying Bomb, Last updated: 24 June 2009.

[2] http://www.centennialofflight.net/essay/Rotary/early\_20th\_century/HE2.htm, 
U.S. Centenial of Flight Commission, Helicopter Development in the Twentieth Century.

[3] ``Robust Adaptive Control Design for Quadcopter Payload Add
and Drop Applications'', Bara J. Emran, Jorge Dias, Lakmal Seneviratne,
Guowei Cai Robotics Institute, Khalifa University, Abu Dhabi, United
Arab Emirates.

[4] ``Autonomous Navigation of Generic Monocular Quadcopter in
Natural Environment'', Kumar Bipin, Vishakh Duggal and K.Madhava
Krishna, Robotics Research Lab, Telengana, India.

[5] ``Cooperative Grasping and Transport Using Multiple Quadrotors'',
Daniel Mellinger, Michael Shomin, Nathan Michael, and Vijay Kumar,
GRASP Laboratory, University of Pennsylvania, Philadelphia.

[6] ``Quad Rotorcraft Control, Vision-Based Hovering and Navigation'',
Luis García, Alejandro Dzul, Rogelio Claude, Centre de Recherches
de Royalieu, Université de Technologie de Compiègne Compiègne cedex,
France.

\newpage

[7] ``Diseño e Implementación de un Sistema Embebido de Control
de actitud para aeronaves no tripuladas'', Ariel Lutenberg, Alan
Kharsansky, Facultad de ingeniería Universidad de Buenos Aires.

[8] ``PID vs LQ Control Techniques Applied to an Indoor Micro
Quadrotor'', Samir Bouabdallah, Andre Noth and Roland
Siegwart, Autonomous Systems Laboratory Swiss Federal Institute of
Technology Lausanne, Switzerland.

[9] ``Basic Helicopter Aerodynamics'', J. Seddon. Blackwell Science, Osney Mead, Oxford, 1996.

[10]  ``Towards Dynamically Favourable Quad Rotor Aerial Robots'', P Pounds, R Mahony and J Gresham, 
Australasian Conference on Robotics and Automation, Canberra, ACT, 2004.

[11] ``Quad Rotorcraft Control Vision-Based Hovering and Navigation'', L García, A Dzul, R Lozano, C Pégard, Springer-Verlag
London 2013.

[12] ``Unmanned Rotorcraft Systems'', Cai, Guowei, Springer-Verlag, London 2011

[13]`` Modelling and Control of Quadcopter'', T. Luukkonen, Aalto University, School of Science, 2011.

[14]`` Modelling of Quadrotor Helicopter Dynamics'', Y.Amir, V. Abbas, National University of Science and Technology, 
Karachi, Paksitan, 2008.

[15] Test Thrust Stand Turnigy'', www.hobbyking.com. 

[16]`` Dynamic Modeling and Intuitive Control Strategy for an X4-flyer '', N. Guenard, T. Hamel, V. Moreau, 2005.

[17]`` Tilt Sensing Using a Three-Axis
Accelerometer '', Mark Pedley, Freescale Semiconductor, 2012-2013.

[18]`` Different Approaches of PID Control UAV Type Quadrotor '', G. Szafranski, R. Czyba, ilesian University of Technology, Poland, 2011.

%Aquí voy, tengo que ver si se cita primero otra cosa.
[19] REFERENCIA DE SINTONIZACIÓN

[20] FILTRO DE KALMAN

[21] FILTRO DE WIENER

	\newpage
\end{document}
