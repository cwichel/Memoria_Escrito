% Soporte de separacion en archivos
\usepackage{subfiles}
% =========================== LAYOUT
% Geometría del documento (tamaños)
\usepackage[
		letterpaper,
		top=2.5cm,
		bottom=2.5cm,
		left=3.5cm,
		right=2.3cm,
		bindingoffset=6mm
		]{geometry}
% Cabeceras/Footers
\usepackage{fancyhdr}
\renewcommand\headrulewidth{0pt}
\renewcommand\footrulewidth{0pt}
\fancyhead[L]{}
\fancyhead[C]{}
\fancyhead[R]{}
\fancyfoot[L]{}
\fancyfoot[C]{}
\fancyfoot[LE,RO]{\thepage}
% =========================== TEXTO
% Soporte de idioma y fuentes
\usepackage[utf8]{inputenc}
\usepackage[spanish]{babel}
\usepackage{
		amsmath, 
		amssymb,
		amstext, 
		gensymb
		}
\usepackage{lipsum}
\usepackage{times}				% Selecciona Times New Roman como fuente
% Interlineado
\usepackage{setspace}
\setstretch{1.5}
% Soporte para listas
\usepackage{enumerate}
% Soporte para hipervinculos
\usepackage[bookmarks=true]{hyperref}
% =========================== FIGURAS/TABLAS/CODIGO
% Permite agregar marcos 
\usepackage{framed}
% Soporte de imagenes
\usepackage{eso-pic}
\usepackage{graphicx}
\graphicspath{{\main/imagenes/}}
% Soporte para flotantes
\usepackage{float}				% Permite usar H como especificador de posición de figuras
% Soporte de codigo
\usepackage{listings}
\lstset{inputpath=\main/codigo/}
% Soporte para figuras
\usepackage{caption}
\usepackage[
		font=footnotesize,
		caption=false,
		farskip=0mm,
		captionskip=0mm,
		nearskip=0mm
		]{subfig}
% Soporte para tablas
\usepackage{array}
\newcolumntype{L}[1]{>{\raggedright\let\newline\\\arraybackslash\hspace{0pt}}m{#1}}
\newcolumntype{C}[1]{>{\centering\let\newline\\\arraybackslash\hspace{0pt}}m{#1}}
\newcolumntype{R}[1]{>{\raggedleft\let\newline\\\arraybackslash\hspace{0pt}}m{#1}}
\usepackage{
		multirow, 
		multicol
		}
% =========================== BIBLIOGRAFIA/GLOSARIO
% Soporte para bibliografia
\usepackage{biblatex}
% Soporte para glosarios
\usepackage{glossaries}
% Permite mantener las marcas al crear el PDF
\usepackage{bookmark}
% =========================== SIN REVISAR
\usepackage{blindtext}
\usepackage{verbatim}
\usepackage{lscape}
\usepackage{alltt}
\usepackage{color} 
\usepackage{soul}
\usepackage{url}
\usepackage{xr}
\definecolor{mygreen}{RGB}{28,172,0} % color values Red, Green, Blue
\definecolor{mylilas}{RGB}{170,55,241}